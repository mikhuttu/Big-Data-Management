% This is "sig-alternate.tex" V2.1 April 2013
% This file should be compiled with V2.5 of "sig-alternate.cls" May 2012
%
% This example file demonstrates the use of the 'sig-alternate.cls'
% V2.5 LaTeX2e document class file. It is for those submitting
% articles to ACM Conference Proceedings WHO DO NOT WISH TO
% STRICTLY ADHERE TO THE SIGS (PUBS-BOARD-ENDORSED) STYLE.
% The 'sig-alternate.cls' file will produce a similar-looking,
% albeit, 'tighter' paper resulting in, invariably, fewer pages.
%
% ----------------------------------------------------------------------------------------------------------------
% This .tex file (and associated .cls V2.5) produces:
%       1) The Permission Statement
%       2) The Conference (location) Info information
%       3) The Copyright Line with ACM data
%       4) NO page numbers
%
% as against the acm_proc_article-sp.cls file which
% DOES NOT produce 1) thru' 3) above.
%
% Using 'sig-alternate.cls' you have control, however, from within
% the source .tex file, over both the CopyrightYear
% (defaulted to 200X) and the ACM Copyright Data
% (defaulted to X-XXXXX-XX-X/XX/XX).
% e.g.
% \CopyrightYear{2007} will cause 2007 to appear in the copyright line.
% \crdata{0-12345-67-8/90/12} will cause 0-12345-67-8/90/12 to appear in the copyright line.
%
% ---------------------------------------------------------------------------------------------------------------
% This .tex source is an example which *does* use
% the .bib file (from which the .bbl file % is produced).
% REMEMBER HOWEVER: After having produced the .bbl file,
% and prior to final submission, you *NEED* to 'insert'
% your .bbl file into your source .tex file so as to provide
% ONE 'self-contained' source file.
%
% ================= IF YOU HAVE QUESTIONS =======================
% Questions regarding the SIGS styles, SIGS policies and
% procedures, Conferences etc. should be sent to
% Adrienne Griscti (griscti@acm.org)
%
% Technical questions _only_ to
% Gerald Murray (murray@hq.acm.org)
% ===============================================================
%
% For tracking purposes - this is V2.0 - May 2012

\documentclass{sig-alternate-05-2015}

\begin{document}

\title{Scalable data cleansing based on qualitative attributes}

\numberofauthors{1}

\author{
\alignauthor
	Mika Huttunen\\
       		\affaddr{Helsinki University}
}

\date{11 March 2018}

\maketitle

%\begin{abstract}
%This is an example abstract...
%\end{abstract}

%
% The code below should be generated by the tool at
% http://dl.acm.org/ccs.cfm
% Please copy and paste the code instead of the example below. 
%
%\begin{CCSXML}
%<ccs2012>
% <concept>
%  <concept_id>10010520.10010553.10010562</concept_id>
%  <concept_desc>Computer systems organization~Embedded systems</concept_desc>
%  <concept_significance>500</concept_significance>
% </concept>
% <concept>
%  <concept_id>10010520.10010575.10010755</concept_id>
%  <concept_desc>Computer systems organization~Redundancy</concept_desc>
%  <concept_significance>300</concept_significance>
% </concept>
%</ccs2012>  
%\end{CCSXML}

%\ccsdesc[500]{Computer systems organization~Embedded systems}
%\ccsdesc[300]{Computer systems organization~Redundancy}

\printccsdesc

\keywords{data cleansing, big data, scalability}


% 1. INTRODUCTION
\section{Introduction}

Nowadays companies are gathering large amounts of data in different ways such as via user input, and all kinds of sensors. It's also becoming easier and easier for them to use data in all kinds of decision making, analytics, and for example in automating human-involved tasks. Problems arise when the used data is \textit{dirty}, or invalid, and thus can lead into making incorrect decisions. These can sometimes cause serious issues - especially in health care and financing sides \cite{Ilyas2015}.

Humans often make data input errors via misspelling, and sensors may grab unwanted noise along the data they're designed to catch. In fact, over $25$\% of critical data in the world's top companies is flawed \cite{Khayyat2015}. Not to mention that today, the variety of data is also large which leads into collecting data of different formats together. \textit{Data cleansing} is the solution to the above-mentioned problem. For data processors, understanding data cleansing, and the problems it tries to solve is thus naturally important. 

Data cleansing is a research field that explores ways to improve \textit{quality} of dirty data. If data isn't of high quality, it means that it has usually both \textit{schema}, and \textit{instance} level problems \cite{Rahm2000}. Another used definition for dirty data is that it doesn't meets its usage needs. A simple example of that would be such that a front developer of a service can't provide end-users a good UI because the server cannot deliver all kinds of useful data because there are faults in the way the data is stored in the database.

Figure \ref{rahm1} shows how \textit{single-source problems} with data quality can yet be divided into schema and instance level data problems. Single-source problems simply stand for such that can occur in data that is stored in a single site with possible data replication on the database side. The same problems can also arise in \textit{multi-source} systems where the data may also be scattered across several sites. %Today, there is no good solution for cleaning huge amounts of data in multi-source systems although some research towards this has been made \cite{Ilyas2015} (p. $379$).

\begin{figure}
	\centering

	\includegraphics[scale=0.8]{figures/rahm1.png}
	\caption{Single-source data quality problems \cite{Rahm2000}} \label{rahm1}
\end{figure}

Instance level problems in data arise as attributes of data \textit{tuples} being out of their scope, or just wrong. These contain misspellings, and \textit{outliers} by for example possible sensor errors, and processing errors before gathered data is actually stored. These problems can be detected via \textit{quantitative techniques} related to data cleansing that focus on error detection and correction based on numerical attributes of data \cite{Hellerstein2008}. If for example some framework can reduce multi-dimensional data in two-dimensional space, it's fairly easy for an expert to detect outliers in that space, and thus possible errors in the data.

Schema level problems on the other hand arise as \textit{violations} of \textit{rules} such as \textit{integrity constaints} (IC), and \textit{functional dependencies} (FD) \cite{Ilyas2015} (p. $289 - 291$). As an example, functional dependency
\begin{equation}
  ZIP \rightarrow STATE \label{eq1}
\end{equation} defines a constraint between data attributes $ZIP$ and $STATE$ such that $ZIP$ implies the $STATE$. In other words, if we had a dataset with two tuples having the same $ZIP$ attribute, but different values of $STATE$, and we trust rule \ref{eq1}, at least one of them is erroneous.

\textit{Duplicate} (or partically duplicate) data entries can also be considered both schema level, and instance level data quality problems \cite{Ilyas2015} (p. $283$). They usually appear as the same entries with some attributes having different values from each other, and thus should be classified as instance level problems. They can however be detected by similar rules that are used for other schema level problems. Consider us having a person register where each person has its name, phone number, address, zip code ($ZIP$), and state ($STATE$) defined. Now we could have two tuples with the same name, phone number, and address, but different zip codes. An error between these two tuples found be detected by rule \ref{eq1}, and could be solved either automatically, or via human interaction. 

This seminar report focuses on highly scalable \textit{qualitative} data cleansing techniques. They focus on detecting data errors in \textit{big data} via rules such as FDs, \textit{CFDs} (conditional functional dependencies), and \textit{DCs} (denial constraints). Ilays and Chu explain these in their paper \cite{Ilyas2015} (p. 287 - 301).

In traditional ways, the data errors are detected by algorithms that compare data tuples trying to find violations based only on user-defined rules, or also such rules or \textit{patterns} that the data cleansing framework itself generates by processing the data. The found errors can afterwards be fixed either via totally automatic, or (partially) human guided procedures.

The rest of the report is divided as follows: in Section $2$, I will introduce recent findings related to big data cleansing, and where we stand on the field at the moment. There should also be a separate section for more accurate introductions to different kinds of state-of-art frameworks (especially HoloClean). Alongside that, I should also have a section for discussing weak points in the papers. So far, I haven't really found anything else than somewhat bad inner organisation of different topics - especially in \cite{Ilyas2015}. And beside them, I would also of course add a section of more detailed analysis of the papers, and what there is left to be improved with data cleansing. As far as I understand, HoloClean is an awesome framework for big data cleansing tasks when all the data is located under a single source, or site. Problems however arise when applying data cleansing for multi-source data. This is especially because the shipment of huge amounts of data over the network is slow \cite{Ilyas2015} (p. $379$). 

%Section $3$ gives background for following sections by introducing different kinds of techniques related to handling big data, but also for getting more accurate data cleansing results. Section $4$ discusses recent frameworks for scalable data cleansing big data, and in Section 5, I have a brief conclusion to summarise the report.
% TODO: section 5: omat ideat, algoritmit yms., conclusion: section 6

% 2. LITERATURE REVIEW
\section{Literature review}

Data cleansing has been an interesting topic for decades, and there has been a large amount of research on the field in the recent years,  Extensive summary from $2015$ by Ilyas and Chu \cite{Ilyas2015} discusses recent techniques on both error detection and repairing of relational data. The errors are generally detected in data as violations of rules such as FDs, CFDs and DCs.

Ilyas and Chu define violation respect to a rule $r$ being the minimal subset of database cells such that at least one of the cells has to be modified to satisfy $r$. Here cell stands for an attribute value of a tuple, or a row. The same definition can of course be extended to non-relational, but structured data that NoSQL document databases can consist of.

The manual construction of data quality rules such as FDs, CFDs and DCs is time-consuming and requires lots of domain expertise. Thus having automatic tools for their generation is essential \cite{Chu2016-2}. Previously implemented $TANE$ and $FASTFD$ that Ilyas and Chu discuss \cite{Ilyas2015} can be used for finding FDs based on a relational data, and its schema, while $FASTDC$ can be used for finding DCs.

%Figure \ref{chu2} represents their classification of qualitative error detection techniques.

%\begin{figure*}
%	\centering
%
%	\includegraphics[scale=0.8]{figures/chu2.png}
%	\caption{Data repairing techniques \cite{Chu2016}} \label{chu2}
%\end{figure*}

The following part of the report is yet to be written out. But the plan was to especially discuss recent frameworks for error detection with relatively small datasets, and then introduce more recent, scalable ones like BigDansing (distributed rule-based error detection and correcting), SampleClean (probability based fixed data sampling while having dirty database on the background), DeDoop / Dis-Dedup (frameworks specifically designed for duplicate tuples detection), and HoloClean. HoloClean is by far the most interesting of the above-mentioned frameworks. This is because it's designed to be applicable for detecting errors that are untrackable by comparing tuples with each other by using only a single rule. 

Abedjan et al. show in their research paper from $2016$ that even though frameworks such as BigDansing are highly scalable, and can be applied to cleansing all kinds of datasets, they can track only approximately $40$\% of all possible errors - even with fully optimised configuration parameters. If an expert is however used to cleanse data by applying several frameworks in the right order for specific data cleansing tasks, much more than $40$\% of the possible errors are trackable, but the cleansing accuracy still hardly reaches even $70$\% \ref{Abedjan2016}.

This specific research probably inspired partially the same research team in designing HoloClean of which purpose was to overcome the long lasted challenges with data cleansing tools not achieving good cleansing accuracy. According to the Rekatsinas et al., HoloClean achieves over twice the accuracy of existing state-of-art methods with average precision of $90$\% \ref{Rekatsinas2017}. What makes HoloClean so much better than the existing state-of-art methods is that accordingly to its name, it's a \textit{holistic} data cleansing framework. Holisticity means that it can do tuple-based comparisons by not just using one rule at the time, but the set of all defined rules.

Holisticity in data cleansing was no way a new innovation. Chu et al. published a research paper on the topic in 2013 \ref{Chu2013}. For now, I haven't done enough research to find out, why HoloClean wasn't invented sooner though. Perhaps the real need for it wasn't really known before $2016$ when Abedjan et al. published their work about cleansing accuracy going significantly up when carefully selecting, which data cleansing frameworks to use, for what tasks and in what order \ref{Abedjan2016}.





% Data cleansing has been an interesting topic for decades \cite{Khayyat2015} although no extensive research existed before the 21st century \cite{rahm2000}. In paper from $2000$ when NoSQL didn't exist yet, Rahm and Do \cite{rahm2000} discuss data cleansing especially as a part of \textit{ETL} process (extraction-transformation-loading). In ETL process, data is extracted from various sources, it's integrated together, and eventually loaded into a data warehose for further processing. The process involves data cleansing as part of both extraction and integration phases such that in extraction phase, each source should be cleansed separately, and in integration phase, data transformations are applied on the data to format the data in a uniform way. On this phase, duplicate records from various sources should also be tracked, and merged together.



% -----------------------
% KERRO N�IST�

% Most data cleansing frameworks involve human interaction: either by training a ML classifier built-in the framework to solve conflicts on its own, or solving conflicts themselves

%Detecting, and removing (partially) duplicate entries is one of the key areas in data cleansing, and frameworks such as \textit{DeDoop} and \textit{Dis-Dedup} which is examined in section $4$ are specifically designed for their detection.

% 2015: TRENDS ja muiden artikkelien l�pik�ynti. Osan asioista voi j�tt�� big datalle tarkoitettuun osioon.
% Selke� aikajatkumo l�hivuosien tapahtumista (ICs, frameworkej� n�iden p��lle, skaalautuminen big datan k�sittelyyn (mm. BigDansing), tarkempaa tutkimusta ty�kalujen tarkkuudesta virheiden korjaukseen (2016: "Where are we now?"), ty�kalujen yhdist�mist� tarkkuutta parantamalla, 2017: "HoloClean"
% 2017: CleanM (miten suhtautuu jatkumoon), 2017: CLAMS (menee jonkin verran asian ohi). Pit�isik� vain mainita, mutta ei menn� kovin tarkasti t�m�n ongelman ratkaisuun? Johtop��t�sten yheydess� voi mainita ongelmasta unstructured datan k�sittelyyn liittyen (samoin kuin se, ett� dataa usealla saitilla jota puhdistaa).

%The rest of the report is will be divided as follows. In section 2, I will summarise, where we are with big data cleansing at the moment. I'll introduce several big data cleansing techniques in more detail. The techniques can practically be divided into \textit{de-duplication} methods, \textit{sampling}, \textit{incremental cleansing}, and \textit{distributed cleansing}. I'll also discuss what kinds of problems are yet to be solved. 

% skaalautuvuus --> oletuksia, dataa usealla 'saitilla', holistic vs. non-holistic cleansing, ... ?
% -----------------------




% 2008
%\textit{Quantitative Data Cleaning for Large Databases}

%- Quantitative data (outlier detection)
%- Categorical data (same thing mentioned with a different name, misspellings)


% 2011
%\textit{An Efficient Data Cleaning Algorithm Based on Attributes Selection}

%- SNM \& MPN algorithms for duplicate records detection
%- Improved algorithm for duplicate error detection
%- O(n log n)
%- not suitable for big data?


%\textit{Qualitative Data Cleaning}

%- What kind of errors, how and where to detect them
%- Data repairing
%-- Trusting integrity constraints / rules, data or both?
%-- Automatic / human guided (training ML model, suggesting fixes etc.)
%-- Repairing on place / generating a model for repairing

%\section{2015: TRENDS}

%- Data deduplication can be seen as enforcing a key constraint defined on all the attributes of
%a relational schema, since two duplicate tuples can be seen as a violation
%of the key constraint.

%\subsection{Error detection}

%- Given a relational database instance I of schema R
%and a set of integrity constraints ?, we need to find another database
%instance I' with no violations with respect to ?.

%- Given a dirty database instance, the first step toward cleaning the
%database is to detect and surface anomalies or errors.

%- Automation (How to Detect?): We classify proposed approaches
%according to whether and how humans are involved in the
%anomaly detection process. Most techniques are fully automatic,
%for example, detecting violations of functional dependencies,
%while other techniques involve humans, for example, to identify
%duplicate records.

%- Let ? denote a set of integrity constraints (ICs). We use I |= ?
%to indicate that database Instance I conforms with the set of integrity
%constraints ?. We say that I' is a repair of I with respect to ? if
%I'  |=  ?, and CIDs(I) = CIDs(I') (i.e., no deletions or insertions are allowed to clean a database instance).

%- Constraints (refer to p. 289 - 301)
%-- FD, CFD, DC (subsumes the former so explaining them not useful?)

%- Duplicate record detection
%-- Similarity graph based on some similarity metric
%-- Clustering enough similar records together (A <--> B and B <--> C    ---->  A <--> C)
%--- In other words, finding connected components in the similarity graph
%-- Merging the records into a single records (based on their attributes)


%Holistic data cleaning (p. 318):
%- Automatically detects violations based on multiple ICs

%Where to detect (p. 322)
%- Problematic that data errors are often found in business layers


%\subsection{Error reparing}

%- Repair target (What to repair?)
%-- Repairing algorithms make different
%assumptions about the data and the quality rules: (1) trusting
%the declared integrity constraints, and hence, only data can
%be updated to remove errors; (2) trusting the data completely
%and allowing the relaxation of the constraints, for example, to
%address schema evolution and obsolete business rules; and finally
%(3) exploring the possibility of changing both the data and the constraints.

%- -Holistic repairing (p. 336 - 342)
%--- We describe these techniques as 'One at a time techniques'. Most available data repairing solutions are in this category.
%They address one type of error, either to allow for theoretical quality guarantees, or to allow for a scalable system.

%--- Multiple data quality problems, such as missing values, typos, the presence
%of duplicate records, and business rule violations, are often observed
%in a single data set. These heterogeneous types of errors interplay
%and conflict on the same dataset, and treating them independently
%would miss the opportunity to correctly identify the actual errors in the
%data. We call the proposals that take a more holistic view of the data
%cleaning process Holistic cleaning approaches. 

%-- Mention also rules-only repairing, and both data and rules repairing (p. 345 - 350)
%--- Continuous data cleansing


%- Automation (How to repair?)
%-- Fully automatic (trying to minimize cost between moving from I to I') vs. human involvement (verify / suggest fixes, train ML model)

%-- Fully automatic: cardinality-minimal and cost-minimal repairs (p. 351)
%--- - Unverified fixes, may introduce new errors during the process

%-- Automatic data repairing techniques use heuristics, such as minimal
%repairs to automatically repair the data in situ, and they often generate
%unverified fixes. Worse still, they may even introduce new errors during
%the process. It is often difficult, if not impossible, to guarantee the
%accuracy of any data repairing techniques without external verification
%via experts and trustworthy data sources. (p. 357)

%-- Example 3.10. Consider two tuples t1 and t8 in Table 1.1; they both
%have the same values ?25813? for ZIP attribute, but t1 has ?WA?
%for ST attribute and t8 has ?WV? for ST attribute. Clearly, at least
%one of the four cells t1[ZIP], t8[ZIP], t1[ST], t8[ST] has to be incorrect.
%Lacking other evidence, existing automatic repairing techniques [17, 26]
%often randomly choose one of the four cells to update. Some of them [17]
%even limit the allowed changes to be t1[ST], t8[ST], since it is unclear
%which values t1[ZIP], t8[ZIP] should take if they are to be changed. (p. 357)

%-- Guided data repair
%-- KATARA (scale-out?)
%-- Data Tamer (scale-out?)


%- Repair model (Where to repair?)
%-- Repair database instance in place vs. generate a model for repairing the instance

%-- Most of the proposed data repairing techniques 
%(all discussed so far) identify errors in the data, and find a unique fix
%of the data either by minimally modifying the data according to a cost
%function or by using human guidance (Figure 3.17(a)). (p. 366)

%-- As follows, we describe a different model-based approach for nondestructive
%data cleaning. Data repairing techniques in this category do
%not produce a single repair for a database instance; instead, they produce
%a space of possible repairs (Figure 3.17(b)). The space of possible
%repairs is used either to answer queries against the dirty data probabilistically
%(e.g., using possible worlds semantics) [12], or to sample
%from the space of all possible clean instances of the database [9, 11].

%-- Probabilistic Cleaning
%--- Probabilistic deduplication (probability based duplication "removal")

%-- Sampling the Space of Possible Repairs
%--- Relates closely to SampleClean for big data cleansing



%BIG DATA CLEANING
%- Pyrkimys v�hent�� tarvittavaa ihmiskommunikaatiota
%- Mahd. vain datan pienen osajoukon k�sittely ja t�m�n perusteella todenn�k�isyyksiin perustuvia vastauksia / jatkok�sittely�

%- Deduplicating
%-- Blocking, windowing, canopy clustering

%- Sampling (SampleClean)

%- Incremental data cleaning
%-- Entity resolution (ER) algorithm

%- Distributed data cleaning
%-- MapReduce, Spark, Dedoop (with ER) / Dis-Dedup, 
%-- BigDansing (runs on top of a data processing platform like DBMS or MapReduce)
%-- HoloClean?
%-- CleanM ?

%- Problems with data partitioned across multiple sites (p. 379)
%-- Objective to minimize data shipment cost



%This section discusses background related to data cleansing.
%It introduces a general process for data cleansing containing
%data analysis and data transformation steps.

%It also introduces specific data quality related problems and
%different kinds of methods and frameworks for handling them.


% 5. BIG DATA CLEANSING
%\section{Big data cleansing}

%This section discusses problems that arise when handling big data.

%- "Scalability. Large volumes of data render most current techniques unusable in real settings. "

%This section also discusses recent frameworks designed particularly for handing big data.

%I did further investigation with BigDansing, SampleClean, HoloClean, Dis-Dedup, and CleanM frameworks.
%Ihab F. Ilyas has been part of research team of BigDansing, HoloClean and Dis-Dedup joined by
%Xu Chu (with the exception of not being part of BigDansing research team) et al. and thus
%I picked also CleanM for further investigation not to only have content from the same researchers.
%Ilyas and Chu's have nevertheless conducted lots of essential research especially related to qualitative data cleansing.

%TODO: I also discuss CLAMS which is not actually a big data cleansing framework, but rather one
%for managing data quality via ICs with limited schema information (?).


%BigDansing is a framework that can be run for example on top of
%MapReduce to distribute error detection work.

%HoloClean is 

%Dis-Dedup is a framework for detecting duplicate data records. % improved version of DeDoop



% 6. WEAK POINTS ON PAPERS
%\section{Weak points on papers}

%This section discusses weak points of the papers related to the
%papers current ongoing situation with big data cleansing.



% 7. OWN IDEAS
%\section{Own ideas}

%Give new ideas/algorithms/experiments on this research problem. This part is very important, because it
%shows the potential of the author to be an independent innovative researcher. This section can be as long as
%possible.

%Come up with a new idea / algorithm that could be used for
%dealing with big data and does something better than the existing
%tools available...



% 8. CONCLUSION
%\section{Conclusion}

%Summarize the research problem and the main contributions of previous papers. The main weakness of
%previous works could be also mentioned here. Some future works can be described as well.



% The following two commands are all you need in the
% initial runs of your .tex file to
% produce the bibliography for the citations in your paper.
\bibliographystyle{abbrv}
\bibliography{sigproc}  % sigproc.bib is the name of the Bibliography in this case
% You must have a proper ".bib" file
%  and remember to run:
% latex bibtex latex latex
% to resolve all references



% 10. EXAMPLES
%\section{Examples}

%\begin{table}
%\centering
%\caption{Frequency of Special Characters}
%\begin{tabular}{|c|c|l|} \hline
%Non-English or Math&Frequency&Comments\\ \hline
%\O & 1 in 1,000& For Swedish names\\ \hline
%$\pi$ & 1 in 5& Common in math\\ \hline
%\$ & 4 in 5 & Used in business\\ \hline
%$\Psi^2_1$ & 1 in 40,000& Unexplained usage\\
%\hline\end{tabular}
%\end{table}

%\begin{figure}
%\centering
%\includegraphics[height=1in, width=1in]{rosette}
%\caption{A sample black and white graphic that has
%been resized with the \texttt{includegraphics} command.}
%\vskip -6pt
%\end{figure}

%\newtheorem{theorem}{Theorem}
%\begin{theorem}
%Let $f$ be continuous on $[a,b]$.  If $G$ is
%an antiderivative for $f$ on $[a,b]$, then
%\begin{displaymath}\int^b_af(t)dt = G(b) - G(a).\end{displaymath}
%\end{theorem}

%\newdef{definition}{Definition}
%\begin{definition}
%If $z$ is irrational, then by $e^z$ we mean the
%unique number which has
%logarithm $z$: \begin{displaymath}{\log e^z = z}\end{displaymath}
%\end{definition}

%\begin{proof}
%Suppose on the contrary there exists a real number $L$ such that
%\begin{displaymath}
%\lim_{x\rightarrow\infty} \frac{f(x)}{g(x)} = L.
%\end{displaymath}
%Then
%\begin{displaymath}
%l=\lim_{x\rightarrow c} f(x)
%= \lim_{x\rightarrow c}
%\left[ g{x} \cdot \frac{f(x)}{g(x)} \right ]
%= \lim_{x\rightarrow c} g(x) \cdot \lim_{x\rightarrow c}
%\frac{f(x)}{g(x)} = 0\cdot L = 0,
%\end{displaymath}
%which contradicts our assumption that $l\neq 0$.
%\end{proof}

\end{document}